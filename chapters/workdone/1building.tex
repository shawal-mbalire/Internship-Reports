\section{Building Computers}

In this section, we delve into a significant project undertaken during our internship that centered on an environmentally conscious initiative: repurposing electronic waste (e-waste) obtained from MTN, a prominent telecommunications company. The project's primary objective was to salvage discarded electronic components and employ them in constructing a fully functional personal computer system. This initiative seamlessly aligns with MTN's steadfast commitment to sustainability, contributing to a reduction in electronic waste deposited in landfills and promoting the reuse of valuable resources.

\subsection{Project Overview}

The project commenced with a meticulous assessment of the collected e-waste, which included components such as motherboards, processors, memory modules, and storage devices. Our team paid careful attention to the compatibility and condition of these components, ensuring that they met the criteria for optimal performance and longevity. The construction process entailed a methodical assembly, starting from fitting the components onto the motherboard to configuring the BIOS settings for seamless operation.

\subsection{Challenges and Creative Problem-Solving}

One of the notable challenges encountered during this project was the significant variation in component specifications and interfaces. This diversity necessitated creative problem-solving to overcome compatibility issues. Collaborative efforts with colleagues and the guidance provided by our mentors proved invaluable in troubleshooting and devising effective solutions to ensure the project's success.

\subsection{Achieving Success}

The successful culmination of the project resulted in the creation of a fully functional personal computer system. This achievement served as a tangible demonstration of the feasibility of sustainable practices and the positive impact they can have on the environment. It also underscored the significance of technical skills, adaptability, and resourcefulness when addressing real-world challenges.

\subsection{Alignment with Sustainable Practices}

This project exemplifies the alignment of our efforts with broader global initiatives aimed at responsible e-waste management. It serves as a testament to the potential for positive outcomes through innovative thinking and collaborative endeavors. Our work not only showcases the practicality of repurposing e-waste but also reinforces the importance of environmentally conscious practices in today's world.

In summary, the "Building Computers" project during our internship at NetLab!UG demonstrates our commitment to sustainability, technical proficiency, and the capacity to address environmental challenges through hands-on, practical solutions.
