\subsection{Building Computer}
This subsection delves into a notable project undertaken during the internship, focusing on the environmentally conscious initiative of repurposing electronic waste (e-waste) obtained from MTN, a prominent telecommunications company. The project aimed to salvage discarded electronic components and utilize them in constructing a functional personal computer system. E-waste management aligns with MTN's commitment to sustainability, contributing to reduced electronic waste in landfills and promoting the reuse of valuable resources.

The project commenced with a thorough assessment of the collected e-waste, including components such as motherboards, processors, memory modules, and storage devices. Careful consideration was given to the compatibility and condition of the components to ensure optimal performance and longevity. The construction process involved meticulous assembly, from fitting the components onto the motherboard to configuring the BIOS settings.

One of the significant challenges encountered was the variation in component specifications and interfaces, necessitating creative problem-solving to overcome compatibility issues. Collaborative efforts with colleagues and guidance from mentors proved invaluable in troubleshooting and devising effective solutions.

The successful completion of the project resulted in a fully functional personal computer system that showcased the potential of repurposing e-waste. This accomplishment not only demonstrated the feasibility of sustainable practices but also highlighted the importance of technical skills, adaptability, and resourcefulness in addressing real-world challenges. The project aligns with broader global efforts towards responsible e-waste management and serves as a testament to the positive outcomes achievable through innovative thinking and collaborative endeavors.
