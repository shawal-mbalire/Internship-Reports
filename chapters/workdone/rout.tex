\subsection{Configuring a network switch with a microtik router and an access point}
This subsection outlines a pivotal aspect of the internship, detailing the setup and configuration of network infrastructure to facilitate efficient access to the NAS system through a fixed IP address. The integration of a Cisco switch, MikroTik router, and local area network (LAN) aimed to optimize connectivity, data transfer, and seamless user experience.

The configuration process commenced with the Cisco switch, where VLANs were established to segregate network traffic efficiently. Ports were assigned to their respective VLANs based on device types and security requirements. Quality of Service (QoS) settings were fine-tuned to prioritize data traffic, ensuring optimal performance for critical applications.

The MikroTik router configuration followed, with a focus on establishing secure and efficient routing protocols. Static routes were configured to enable seamless communication between different network segments. Firewall rules were implemented to safeguard the network against unauthorized access, while port forwarding was set up to redirect incoming traffic to the NAS system's fixed IP address.

For LAN configuration, IP address assignments were managed through Dynamic Host Configuration Protocol (DHCP) to simplify device connectivity. A dedicated subnet was allocated for the NAS system, ensuring consistent communication and efficient data transfer between devices and the NAS.

The success of the configuration was gauged through thorough testing. Network connectivity, data transfer speeds, and NAS access through the fixed IP address were evaluated. Collaborative efforts among team members facilitated troubleshooting and fine-tuning of configurations.

Challenges encountered during the project included optimizing QoS settings for varying network loads, addressing IP conflicts, and ensuring seamless communication between different segments of the network. These challenges were surmounted through meticulous configuration adjustments and collaboration among team members.

The successful outcome of the project resulted in a robust network infrastructure, enabling efficient access to the NAS system through a fixed IP address. This accomplishment demonstrated the significance of effective network design and configuration in facilitating seamless data transfer, optimizing performance, and enhancing user experience.
