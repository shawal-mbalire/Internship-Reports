\section{Building a truenas server}
This subsection highlights a significant project undertaken during the internship the construction of a TrueNAS server using unconventional materials and advanced configuration techniques. The objective was to create a functional storage server utilizing a cardboard case, implementing a robust RAID Z2 configuration to ensure data redundancy and fault tolerance.

The project commenced with the selection and procurement of appropriate hardware components, including hard disks, a RAID controller, and network interfaces. The unconventional choice of a cardboard case was driven by sustainability considerations, aligning with the organization's commitment to eco-friendly practices.

The RAID Z2 configuration was chosen to strike a balance between data protection and storage capacity. This configuration leverages two parity drives to guard against the simultaneous failure of two disks, ensuring data integrity even in challenging scenarios. The configuration process entailed meticulous setup of the RAID controller and the careful selection of disk drives to ensure compatibility and optimal performance.

The construction process involved assembling the hardware components within the cardboard case while ensuring proper ventilation and heat dissipation. Challenges such as the material's limited durability and the need for creative cable management were addressed through innovation and practical problem-solving.

After the hardware assembly, the focus shifted to software setup. TrueNAS, a powerful open-source network-attached storage (NAS) solution, was installed and configured to manage the RAID array, network connectivity, and user access permissions. The server's performance and stability were rigorously tested, including scenarios simulating disk failures and recovery procedures.

The successful completion of the project resulted in a fully functional TrueNAS server housed in a cardboard case. This achievement not only showcased the fusion of technical expertise and environmental consciousness but also highlighted the versatility of unconventional materials in technology projects. The RAID Z2 configuration and disk selection ensured data reliability and availability, demonstrating the importance of robust storage solutions in a data-driven era.

