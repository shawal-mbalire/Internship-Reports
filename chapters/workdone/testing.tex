\subsection{Testing E-Waste for Functional Components}

This subsection focuses on a significant aspect of the e-waste repurposing project undertaken during the internship: the meticulous testing of collected electronic components to identify functional elements suitable for reuse. The process involved a systematic assessment of components sourced from MTN's electronic waste, ensuring that only viable and operational parts were integrated into the construction of the personal computer system.

The testing phase began with the careful disassembly of discarded electronic devices to extract components such as processors, memory modules, graphics cards, and storage drives. Each component underwent a series of diagnostic tests to ascertain its functionality and condition. These tests included power supply checks, voltage measurements, and connectivity assessments using appropriate testing tools and equipment.

Components that exhibited signs of damage, wear, or malfunction were documented and set aside for proper disposal or recycling. Only components that met the required operational standards were considered for inclusion in the personal computer assembly. Compatibility between components was also a crucial consideration to ensure a seamless and reliable system.

The testing process was not without its challenges. Variability in component types, brands, and ages necessitated the use of a diverse set of testing methodologies. Collaborative efforts among team members, coupled with the guidance of experienced mentors, played a pivotal role in addressing these challenges effectively.

The successful testing of e-waste for functional components not only ensured the quality and reliability of the constructed personal computer but also highlighted the significance of careful assessment in sustainable electronics reuse. This phase of the project underscored the importance of technical skills, attention to detail, and the role of ethical e-waste management practices in the pursuit of environmentally responsible initiatives.
