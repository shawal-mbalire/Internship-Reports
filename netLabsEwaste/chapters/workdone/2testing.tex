\section{Testing E-Waste for Functional Components}

This subsection offers a detailed insight into a crucial facet of the e-waste repurposing project undertaken during our internship: the systematic and meticulous testing of collected electronic components to identify functional elements suitable for reuse. This phase of the project was instrumental in ensuring that only viable and operational parts from MTN's electronic waste inventory were integrated into the construction of the personal computer system.

\subsection{The Testing Process}

The testing phase commenced with the careful disassembly of discarded electronic devices, a process that allowed us to extract a plethora of components, including processors, memory modules, graphics cards, and storage drives. Each component, upon extraction, underwent a testing by plugging into a minimal system to confirm functionality. This process was repeated for every component, ensuring that each one was thoroughly tested and evaluated.

Components that exhibited any signs of damage, wear, or malfunction were meticulously documented and set aside for proper disposal or recycling, in strict adherence to ethical e-waste management practices. Conversely, components that unequivocally met the stringent operational standards we had set were considered prime candidates for inclusion in the personal computer assembly. Compatibility between selected components was another pivotal consideration to ensure a seamless and reliable system.

\subsection{Challenges and Adaptability}

The testing process presented its own set of challenges, primarily stemming from the remarkable variability in component types, brands, and ages found within MTN's electronic waste collection.The systems would range from the 90s Toshiba monitors to 2010's dell monitors. All ram modules ranging from DDR! through DDR4. This inherent diversity necessitated the utilization of a diverse set of testing methodologies. However, these challenges were met with a spirit of collaboration among team members and the invaluable guidance provided by experienced mentors. This collective effort proved instrumental in overcoming testing hurdles effectively and efficiently.

\subsection{Success and Implications}

The successful testing of e-waste for functional components was pivotal in ensuring the quality and reliability of the constructed personal computer system. Beyond this immediate accomplishment, this phase of the project underscored several critical lessons. It emphasized the importance of technical skills, meticulous attention to detail, and the significant role of ethical e-waste management practices in the pursuit of environmentally responsible initiatives.

In conclusion, the "Testing E-Waste for Functional Components" phase of our internship project not only contributed to the overall success of the e-waste repurposing initiative but also served as a testament to our commitment to sustainability and responsible electronics reuse. It reinforces the essential role that conscientious assessment plays in the realm of environmentally responsible practices, highlighting the broader implications of our work.




\section{Attempting PS2 Keyboard Conversion to USB Keyboard}

This section documents an informative learning experience during the internship involving the conversion of a PS2 keyboard to a USB keyboard. The primary goal was to explore the conversion process and understand the complexities involved in making legacy PS2 keyboards compatible with modern USB interfaces.

\subsection{Project Objectives}

The project's primary objectives included:

\begin{itemize}
  \item Exploring the inner workings of PS2 keyboards and their communication protocols.
  \item Investigating the feasibility of converting PS2 keyboard signals to USB signals.
  \item Identifying potential challenges and limitations in the conversion process.
\end{itemize}

\subsection{Technical Exploration}

The project began with a thorough exploration of the PS2 keyboard's internal wiring and protocol. The goal was to understand how PS2 keyboards communicate with computer systems and the unique challenges posed by their signal format.

\subsection{Conversion Attempt}

A conversion attempt was made by soldering the PS2 keyboard's D- and D+ wires to a USB connector. However, it became apparent that this straightforward approach would not suffice. The conversion process only provided power to the PS2 keyboard but did not enable it to communicate effectively with USB-enabled devices. This outcome highlighted the complexities of signal conversion and compatibility between different interfaces.

\subsection{Key Learnings}

Despite the conversion's lack of success, the project yielded valuable insights and key learnings:

\begin{itemize}
  \item Enhanced understanding of PS2 keyboard communication protocols.
  \item Recognition of the challenges in converting legacy hardware to modern interfaces.
  \item Appreciation for the intricacies of signal conversion and compatibility.
\end{itemize}

\subsection{Conclusion}

While the conversion attempt did not achieve the intended result, it was a valuable learning experience that deepened understanding of keyboard protocols and the complexities involved in interface conversions. This project emphasized the importance of careful planning and technical expertise when dealing with legacy hardware and signal conversion challenges.
