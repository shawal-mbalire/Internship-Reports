\section{Exploring Connecting the NAS to the Internet}

This section delves into an intriguing project undertaken during the internship – the exploration of connecting the Network-Attached Storage (NAS) system to the internet. The primary objective was to investigate the feasibility, security implications, and potential benefits of making the NAS accessible remotely over the internet and create a local hosted cloud where users can access their files from anywhere.

\subsection{Project Goals}

The project's main goals included:

\begin{itemize}
  \item Assessing the technical requirements for internet connectivity of the NAS.
  \item Evaluating the security considerations associated with remote access.
  \item Exploring the advantages and use cases of having remote access to the NAS.
\end{itemize}

\subsection{Technical Exploration}

The project initiated with a comprehensive technical exploration of the NAS system and its compatibility with internet connectivity. This involved:

\begin{itemize}
  \item Examining the NAS's network capabilities and available ports.
  \item Researching secure protocols and encryption methods for remote access.
  \item Assessing the hardware and software prerequisites for internet connectivity.
\end{itemize}

\subsection{Exploring Remote Access Solutions}

Initially, we attempted to use a load balancer service from freeloadbalancer.com to forward our Nextcloud server through the internet. However, full control required a monthly subscription fee, which was not in line with our project's objectives.

Subsequently, we explored the option of DNS port forwarding over a MikroTik router to a free domain name provided by noip.com. While this approach appeared promising, we encountered technical challenges that prevented us from establishing the connection.

\subsection{Technical Challenges and Learning}

The project became a valuable learning journey about network firewalls and network administration. Challenges included:

\begin{itemize}
  \item The complexities of load balancer configurations.
  \item The technical intricacies of DNS port forwarding.
  \item The need for a deeper understanding of network security and administration.
\end{itemize}

\subsection{Outcome}

Despite the initial attempts, we ultimately decided to host the Nextcloud server locally due to the technical complexities and subscription costs associated with external services. This local setup allowed for convenient access within the organization's network, although it limited remote accessibility.

In conclusion, the project provided valuable insights into network administration, security, and remote access solutions. While we did not achieve our initial goal of connecting the NAS to the internet, the experience gained contributes to our understanding of network complexities and informs future decision-making regarding remote access solutions.

\section{3D Design of a Computer Case}

This section highlights a hands-on and creative project undertaken during the internship – the design of a computer case using two distinct software tools, Fusion 360 and Blender. The primary objective was to leverage the capabilities of both software platforms to conceptualize, model, and refine a unique computer case design that places a strong emphasis on functionality, aesthetics, and manufacturability.

\subsection{Project Initiation}

The design journey commenced with the utilization of Autodesk Fusion 360, a robust parametric modeling tool. Initial sketches and design concepts were transformed into a digital format, allowing for dynamic adjustments and iterative refinement. The parametric nature of Fusion 360 facilitated the exploration of various design elements, including dimensions, angles, and features, all while ensuring a consistent and coherent design.

\subsection{Enhancing Aesthetics in Blender}

Once the foundational design was established in Fusion 360, the project transitioned to Blender, a powerful 3D modeling and animation software. Blender's versatile toolset enabled the addition of intricate details, textures, and surface finishes that significantly contributed to the computer case's visual appeal. The non-linear workflow of Blender allowed for artistic freedom in refining the aesthetics and adding fine details to the design.

\subsection{Overcoming Challenges}

Throughout the design process, several challenges were encountered. These challenges included optimizing the design for manufacturability, ensuring proper ventilation, and maintaining structural integrity. Addressing these challenges required meticulous design adjustments, creative problem-solving, and collaborative consultation with mentors and peers.

\subsection{Iterative Refinement}

The iterative nature of the design process involved multiple rounds of feedback and refinement. Collaborative efforts were invested in critiquing and enhancing the design to meet both functional and visual objectives. The utilization of both Fusion 360 and Blender proved to be instrumental in visualizing and effectively communicating design concepts.

\subsection{Successful Outcome}

The project culminated in the creation of a comprehensive and visually appealing hexagon computer case design. The integration of Fusion 360's parametric capabilities and Blender's artistic freedom resulted in a design that elegantly balances functionality and aesthetics. This project underscores the significance of leveraging diverse software tools to transform creative visions into tangible and refined designs.

This hands-on experience in 3D design not only expanded technical skills but also provided valuable insights into the intersection of art and engineering within the context of product design.
