\section{Building a TrueNAS Server}

This subsection shines a spotlight on a substantial project that stood out during our internship: the construction of a TrueNAS server using unconventional materials and advanced configuration techniques. The primary objective of this endeavor was to engineer a fully functional storage server within the constraints of an eco-conscious approach, involving the utilization of a cardboard case and the implementation of a robust RAID Z2 configuration to ensure data redundancy and fault tolerance.

\subsection{Project Inception}

The project's inception was marked by careful planning and resource procurement. We meticulously selected and acquired the necessary hardware components, including four hard disks and network interfaces. Notably, the unconventional choice of a cardboard case was driven by a commitment to sustainability. This aligns seamlessly with the organization's steadfast dedication to eco-friendly practices, fostering an awareness of the environmental impact of technology projects.

\subsection{Robust RAID Z2 Configuration}

The heart of the project lay in the RAID Z2 configuration, a choice that aimed to strike an ideal balance between data protection and storage capacity\cite{yi2023optimizations }. This configuration relies on two parity drives to provide a robust shield against the simultaneous failure of two disks, ensuring data integrity even in the face of challenging scenarios. The configuration process was a meticulous endeavor, involving the setup of the RAID controller and a thoughtful selection of disk drives to ensure compatibility and the attainment of optimal performance.

\subsection{Innovative Assembly and Practical Problem-Solving}

The construction phase presented unique challenges, given the unconventional cardboard case. Our innovative approach encompassed the assembly of hardware components while taking into account vital considerations such as proper ventilation and effective heat dissipation. Challenges associated with the material's limited durability and the need for creative cable management were addressed with ingenious solutions. This phase was a testament to our adaptability and practical problem-solving abilities.

\subsection{Software Setup and Rigorous Testing}

Following the successful assembly of hardware, the focus shifted to software setup. TrueNAS, a robust open-source network-attached storage (NAS) solution, was installed and configured to manage the RAID array, network connectivity, and user access permissions. Rigorous testing followed, encompassing scenarios simulating disk failures and recovery procedures, thus ensuring the server's performance, stability, and data reliability in real-world situations.

\subsection{Project Outcome and Lessons Learned}

The culmination of this project bore fruit in the form of a fully functional TrueNAS server, housed within a cardboard case. This achievement transcended mere technical prowess; it underscored the successful fusion of technical expertise with a profound sense of environmental consciousness. Moreover, it served as a powerful reminder of the versatility of unconventional materials in the realm of technology projects.

The RAID Z2 configuration, coupled with meticulous disk selection, stood as a testament to the significance of robust storage solutions in an era characterized by an ever-growing dependence on data. The project was a profound learning experience, highlighting the intrinsic value of innovative thinking, sustainability-driven practices, and the successful execution of advanced technology configurations within the constraints of eco-conscious objectives.
