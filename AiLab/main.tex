\documentclass[a4paper,12pt]{report}

\usepackage[top=2cm,bottom=2cm,right=2cm,left=2cm]{geometry}
\usepackage{datetime}

\title{Report on Communications Team Lead's activities for Makerere AI Lab's Computer Vision for Agriculture Hackathon 2024}
\author{Shawal Mbalire}
\date{\today}

\begin{document}
    \maketitle

    \section{Introduction}    
    
    This hackathon aims to provide undergraduate and postgraduate students from all universities in Uganda with the opportunity to work on computer vision tasks such as image classification and object detection. The participants will have the chance to work with some of the cutting-edge datasets collected under lacuna-funded projects. 
    
    Ahead of the hackathon, I (Shawal Mbalire) was appointed as the lead of the communication team to facilitate the communication before, during and after the hackathon.\\

    \section{Communication Strategy}
    The project begun with formulatinga plan for the communications team which included key activitiesas included in the Hackathon Gannt chart such as;
    \begin{enumerate}
        \item Volunteer Search
        \item Website
        \item Registration process Management
        \item Whatsapp group for questions
        \item Email Reminders
        \item Post hackathathon events among others
    \end{enumerate}
    Three social media streams were formed such as Youtube, LinkedIn and twitter each managed by a volunteer for the hackathon.
    The email messages were to be managed by Me, Judith and Gilbert.

    \section{Volunteer Recruitment}
    % Explain the strategy used to recruit volunteers for communication spread.
    Judith Amutuhire Volunteered to assist me as deputy communications lead to manage the other volunteers and share communication roles.

    A call for volunteer was shared to the different student whatsapp groups such as Datasceince Africa, Developer students community, IEEE and Whatsapp status where volunteers expressed interest.
    
    An online meeting was held to get the volunteer up to date and discus the distribution of roles such as managing the social media handles, website development, graphics design and generally increasing reachout.

    \section{Volunteer Management}
    
    Before registration begun, progress wa stracked of volunteers that had clear tasks assigned to them. The others would only join in on the voluntters meeting to track the progress of the timeline.
    
    The moment registration begun, analytics such as registrants per University were used to track performance by  the volunteers. However, at the 50 volunteer mark, an inner volunteer competetion was kicked off and a new question was added to he form where new registrants would chose whomever refered them. This encouraged competetion among the volunteers and led to a further surge in registrants. 

    A large reason for a high entry rate was the assurance to registrants that there will be learning opportunities from experts. 
    
    A little friction was experienced as some of the volunteer roles disapeared after meetings with the technical team such as the graphics role, and instagram maintainer role. However as all voluteer had the ultimate role of sharing information, this was easily overcome.

    Though a more pressing challenge of forthcoming exams for most universities wasnot covered and some work like youtube Management was shifted to me.


    \section{Website Development}
    I created the website, persented it to the meetings and had it refined over the course of a few meetings. The website was written in html and css and published using firebase free hosting tier.

    The website gives more infromation about the hackathon and links to the registration form. It also has a countdown timer to the proposed start and end of the hackathon

    However soon after the webbsite was done, another webpage with roughly simlar functionality was created by the AI lab which significantly reduced traffic to the creaed wesite.

    \section{Impact and Results}
    As of \today , the registration form has been filled by 170 people, with universities ranging from Lira, Muni to Kabale and Busitema Universities.
    
    % Share feedback received from participants, sponsors, or other stakeholders regarding the communication efforts.

    \section{Lessons Learnt}
    
    Sharing over student whatsapp groups seemed to be the most effective mode of communication thus investing in sharing methods through student leaders would need to take uo most of the communication team resources.
    
    However if the response on important decisions were to be given early, and the technical committee kept a running document of any and all changes of the hackathon, there is a possibility for improvement in communication.
    

    \section{Conclusion}
    % Summarize the key findings and outcomes of the report.
    % Reiterate the significance of effective communication in the success of the hackathon.
    % Thank the volunteers and everyone involved in making the communication efforts a success.
    With the launch of the hackathon gearing up this Saturday, The role of the hackathon reduces but remains to ensure effective feedback and learning among the students participating.

    Special thanks to all the volunteers for the invested time and a special thanks to Judith Amutuhire for supporting as Deputy Lead. 

\end{document}